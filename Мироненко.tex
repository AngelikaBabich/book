\documentclass[12pt, a4paper]{article}
\usepackage[utf8]{inputenc}
\usepackage[ukrainian]{babel}
\usepackage{graphicx}
\graphicspath{ {images/} }
\usepackage{float}
\usepackage{multicol}
\usepackage{blindtext}

\title{Лабораторна робота 1.
	
	ТАНЦЮВАЛЬНА ВЕЧІРКА.}
\author{Розробив вчитель інформатики Мироненко В.А.}
\date{Тривалість - 2 уроки/ 90 хвилин}

\begin{document}
	
	\maketitle
	\begin{abstract}
	Знайомство з роботом mBot. Середовище програмування mBlock. Програмування рухів.
	\end{abstract}
	
	\begin{center}
		\begin{tabular}{|p{0.45\textwidth} | p{0.45\textwidth}|}
			\hline 
			\multicolumn{2}{|c|}{Основна інформація}\\
			\hline
			Тематика & Кольоровий світ  \\ 
			\hline
			Тривалість & 2 уроки/ 90 хвилин \\ 
			\hline
		\end{tabular}
	\end{center}
	
	
	\subsection*{\textit{Цілі:}}
	
	У цьому розділі слухачі повинні:
	\begin{enumerate}
		\item Отримати базові уявлення про програмне забезпечення mBlock.
		\item Вивчити основні компоненти робота mBot.
		\item Зібрати робота та навчитися керувати його рухами.
	\end{enumerate}

	\subsection*{\textit{Ключові моменти:}}
	\begin{enumerate}
		\item Правильно зібрати компоненти робота mBot.
		\item Навчитися використовувати команди програмування руху.
		\item Засвоїти порядок і структуру в програмуванні.
	\end{enumerate}

	\section*{\textbf{Крок 1. Зацікавлення.}}
	\begin{enumerate}
		\item Доктор Панда отримав листа від своєї доброї подруги Доктора Лінди, яка запросила Доктора Панду відвідати благодійний бал у її школі. Мета благодійного балу – зібрати гроші на будівництво більш безпечних шкіл для учнів у віддалених районах. На знак подяки всім, хто пожертвував кошти, вони запланували подарувати захоплюючий танцювальний виступ.
		\item Запитайте учнів, які види танцю вони дивилися.
		
		\textit{Дозвольте учням вільно висловлюватися.}
		\item Запитайте учнів, які танці були найцікавішими.
		
		\textit{Дозвольте учням вільно висловлюватися.}
	\end{enumerate}

	\section*{\textbf{Крок 2. Розвідка.}}
	\begin{enumerate}
		\item Запропонуйте учням переглянути цікаві відеоролики з танцями роботів. Після перегляду відео запитайте учнів: "Як можна охарактеризувати танцювальні рухи цих роботів у відео?".
		
		\textit{Наприклад: уніфіковані та симетричні танцювальні рухи тощо.}
		\item Запитайте учнів: "Що нам потрібно спочатку, щоб виконати танець роботів?"
		
		\textit{Поясніть учням: "По-перше, нам потрібен робот, який може записувати танцювальні рухи".}
		\item Запитайте учнів: "Що ще потрібно, якщо у нас є робот, який може записувати танцювальні рухи?"
		
		\textit{Скажіть учням, що робот повинен вивчити основні танцювальні рухи.}
		\item Попросіть учнів побудувати робота mBot.
	
	\end{enumerate}
		
\begin{figure} [H]
	\centering
	\includegraphics[width=0.25\linewidth]{"E:/Downloads/assembling mbot video"}
	\caption[Відеоінструкція збирання mBot]{Відеоінструкція збирання mBot}
	\label{fig:assembling-mbot-video}
\end{figure}

	\section*{\textbf{Крок 3. Пояснення.}}
\begin{enumerate}
	\item Після того, як всі учні закінчать збірку робота mBot, перевірте всіх роботів відповідно до інструкції. Потім презентуйте цих роботів та ознайомте з кожною їх частиною.
	
	\begin{figure} [H]
		\centering
		\includegraphics[width=1.0\linewidth]{"E:/Downloads/mbot"}
		\caption[Складові mBot]{Складові mBot.}
		\label{fig:mbot}
	\end{figure}
	
	\item Ознайомте учнів з програмним забезпеченням для графічного програмування mBlock.
	
	\textit{Скажіть їм, що за допомогою програмного забезпечення mBlock ми можемо зробити програмування роботів швидким і простим.}

	\item Зверніться до інструкції та попросіть учнів навчитися під'єднувати робота до комп'ютера за допомогою USB-кабелю.
	
	\begin{multicols}{2}
	\textit{
		(1) Відкрийте програмне забезпечення mBlock. Необхідно підключити та встановити драйвер Arduino для першого підключення.
		(2) Підключіть mCore до комп'ютера за допомогою USB-кабелю та увімкніть живлення робота mBot.
		(3) В інтерфейсі програмного забезпечення натисніть на "Connect" і виберіть "Serial Port".
		(4) Потім натисніть на "Плати" в інтерфейсі програмного забезпечення і виберіть "mBot".
		(5) Натисніть на "Connect" в інтерфейсі програмного забезпечення та виберіть "Upgrade Firmware".
		(6) Після успішного підключення можна приступати до програмування.}
	\end{multicols}

	\item Ознайомте учнів з інтерфейсом програми mBlock.
	
		\begin{figure} [H]
		\centering
		\includegraphics[width=1.0\linewidth]{"E:/Downloads/mblock interface"}
		\caption[Інтерфейс mBlock]{Інтерфейс mBlock.}
		\label{fig:mblock-interface}
	\end{figure}
	
	\item Попросіть учнів уважно спостерігати за демонстрацією вчителем операцій програмування та поясніть учням, як перетягувати командні блоки для завершення програмування та як видаляти командні блоки.
	
\end{enumerate}

\section*{\textbf{Крок 4. Розробка.}}
\begin{enumerate}
	\item Попросіть учнів завершити програмування згідно зі схемою програмування, і нехай робот mBot зробить "V-подібний крок".
	
	\begin{figure} [H]
		\centering
		\includegraphics[width=0.5\linewidth]{"E:/Downloads/mbot v step"}
		\caption[V-подібний крок mBot.]{V-подібний крок mBot.}
		\label{fig:mbot-v-step}
	\end{figure}
	
	\textit{Після завершення програмування:
		Натисніть на зелений прапорець у верхньому правому куті сценічного майданчика, щоб запустити програму.
		Натисніть на червону крапку поруч із зеленим прапорцем, щоб зупинити програму.}
	
	\textbf{Важливо}
	
	\textit{Учні повинні повністю розуміти, як керувати напрямками руху роботів.}
	\item Запитайте учнів: "Чи розумієте ви, як керувати траєкторією руху робота mBot?".
	
	\textit{Ви можете отримати такі відповіді: Можу, навчився керувати або нерозумію.}
	
	\item Чи можете ви змусити робота mBot зробити "М-крок"? Спробуйте виконати завдання з програмування самостійно.
	
	\textbf{Важливо}
	
	\textit{Потрібно повністю пояснити кожен крок програми.}
	
\end{enumerate}

\section*{\textbf{Крок 5. Оцінка.}}
\begin{enumerate}
	\item Попросіть студентів заповнити Форму самооцінки. Чи можуть вони це зробити:
	
	
		\begin{itemize}
			\item \textit{Підключити робота до комп'ютера за допомогою USB-кабелю.}
			\item \textit{Виконати основні операції з програмним забезпеченням mBlock.}
			\item \textit{Керувати траєкторією руху робота mBot.}
			\item \textit{Самостійно виконувати базові завдання з програмування.}
		\end{itemize}
	\item Попросіть учнів поділитися своїми думками:
	
	\textit{Які ще танцювальні рухи, на вашу думку, може виконувати робот mBot?
		Які танцювальні рухи були б цікавішими?}

\end{enumerate}

\subsection*{\textit{Домашнє завдання.}}
Самостійно запрограмувати та розробити танцювальні рухи робота mBot, а також спробувати включити музику в танці. Зняти танці на мобільний телефон та надіслати відеоматеріал вчителю.

	
\end{document}